\documentclass[11pt]{article}
\usepackage{amsmath}
\usepackage{amssymb}
\usepackage{amsfonts}

\chapter{Notations}

A large part of robotics is about developing machines that perceives and
interact with the environment. For that robots use sensors to collect and
process data, and knowing what the data describes is of utmost importance.
Imagine obtaining the position of the robot but not knowing what that position
is with respect to. Missing data descriptions such as what a position vector is
expressing, what is it with respect to and more causes many hours of painful
trail and error to extract that information.

In the following section the notation used throughout this document will be
described, and it follows closely of that of Paul Furgale's~\cite{Furgale2014}.
The aim is to mitigate the ambiguity that arises when describing robot poses,
sensor data and more.

A vector expressed in the world frame, $\frame_{\world}$, is written as
$\KineNotationBare{\pos}{\world}$. Or more precisely if the vector describes
the position of the camera frame, $\frame_{\cam}$, expressed in
$\frame_{\world}$, the vector can be written as $\Pos{\world}{\cam}}$ with
$\world$ and $\cam$ as start and end points. The left hand subscripts indicates
the coordinate system the vector is expressed in, while the right-hand
subscripts indicate the start and end points. For brevity if the vector has
the same start point as the frame to which it is expressed in, the same vector
can be written as $\KineNotationPart{\pos}{\world}{\cam}$. Similarly a
transformation of a point from $\frame_{\cam}$ to $\frame_{\world}$ can be
represented by a homogeneous transform matrix, $\Tf{\world}{\cam}$, where its
rotation matrix component is written as $\Rot{\world}{\cam}$ and the
translation component written as $\Trans{\world}{\cam}$. A rotation matrix that
is parametrized by quaternion $\quat_{\world\cam}$ is written as
$\rot\{\quat_{\world\cam}\}$.



\begin{align*}
  &\text{Position:} \enspace & \Pos{\world}{\body} \\
  &\text{Velocity:} \enspace & \Vel{\world}{\body} \\
  &\text{Acceleration:} \enspace & \Acc{\world}{\body} \\
  &\text{Angular velocity:} \enspace & \AngVel{\world}{\body} \\
  &\text{Rotation:} \enspace & \Rot{\world}{\body} \\
  &\text{Transform:} \enspace & \Tf{\world}{\body} \\
  &\text{Point:} \enspace & \Pt{\world}
\end{align*}


\begin{document}


\section{Introduction}


\section{Nonlinear Least Squares}

\begin{align}
  \min_{\Vec{x}} \cost(\Vec{x})
    &=
      \dfrac{1}{2}
      \sum_{i}
      \Vec{e}_{i}^{\transpose} \Mat{W} \Vec{e}_{i} \\
    &=
      \dfrac{1}{2} \enspace
      \Vec{e}_{i}^{\transpose}(\Vec{x})
      \Mat{W}
      \Vec{e}_{i}(\Vec{x})
\end{align}

where the error function, $\Vec{e}(\cdot)$, depends on the optimization
parameter, $\Vec{x} \in \real^{n}$. The error function,
$\Vec{e}(\cdot)$, has a form of
\begin{align}
  \Vec{e}_{i} =
    \Vec{z} - \Vec{h}(\Vec{x})
\end{align}
is defined as the difference between the measured value, $\Vec{z}$, and
the estimated value calculated using the measurement function,
$\Vec{h}(\cdot)$.  Since the error function, $\Vec{e}(\Vec{x})$, is
non-linear, it is approximated with the first-order Taylor series,
\begin{align}
  \Vec{e}(\Vec{x})
    \approx
      \Vec{e}(\bar{\Vec{x}}) +
      \Mat{E}(\bar{\Vec{x}}) \Delta\Vec{x}
\end{align}
where
\begin{align}
  \Mat{E}(\bar{\Vec{x}}) =
    \dfrac{\partial\Vec{e}(\Vec{x})}{\partial\Vec{x}}
    \bigg\rvert_{\Vec{x}_{k}}
\enspace \enspace
  \Delta{\Vec{x}} = \Vec{x} - \bar{\Vec{x}}
\end{align}
\begin{align}
  \dfrac{\partial{\cost}}{\partial{\Vec{x}}} =
    \dfrac{\partial{\cost}}{\partial{\Vec{e}}}
    \dfrac{\partial{\Vec{e}}}{\partial{\Vec{x}}}
\end{align}

\begin{align}
  \dfrac{\partial{\cost}}{\partial{\Vec{e}}} &=
    \dfrac{1}{2} \Vec{e}^{\transpose}(\Vec{x}) \Mat{W} \Vec{e}(\Vec{x}) =
    \Vec{e}^{\transpose}(\Vec{x}) \Mat{W} \\
  %
  \dfrac{\partial{\Vec{e}}}{\partial{\Vec{x}}} &=
    \Vec{e}(\bar{\Vec{x}}) +
    \Mat{E}(\bar{\Vec{x}}) \Delta\Vec{x} =
    \Mat{E}(\bar{\Vec{x}})
\end{align}

\begin{align}
  \dfrac{\partial{\cost}}{\partial{\Vec{x}}}
    &=
      (\Vec{e}^{\transpose}(\Vec{x}) \Mat{W}) (\Mat{E}(\bar{\Vec{x}})) \\
    % Line 2
    &=
      (
        \Vec{e}(\bar{\Vec{x}}) + \Mat{E}(\bar{\Vec{x}}) \Delta\Vec{x}
      )^{\transpose} \Mat{W}
      \Mat{E}(\bar{\Vec{x}}) \\
    % Line 3
    &=
      \Vec{e}^{\transpose}(\bar{\Vec{x}}) \Mat{W} \Mat{E}(\bar{\Vec{x}})
      + \Delta\Vec{x}^{\transpose}
        \Mat{E}(\bar{\Vec{x}})^{\transpose} \Mat{W} \Mat{E}(\bar{\Vec{x}})
      = 0 \\
    % Line 4
    \Delta\Vec{x}^{\transpose}
      \Mat{E}(\bar{\Vec{x}})^{\transpose} \Mat{W} \Mat{E}(\bar{\Vec{x}})
    &=
      - \Vec{e}^{\transpose}(\bar{\Vec{x}}) \Mat{W} \Mat{E}(\bar{\Vec{x}}) \\
    % Line 5
    \underbrace{
      \Mat{E}(\bar{\Vec{x}})^{\transpose} \Mat{W} \Mat{E}(\bar{\Vec{x}})
    }_{\Mat{H}}
      \Delta\Vec{x}
    &=
    \underbrace{
      - \Mat{E}(\bar{\Vec{x}})^{\transpose} \Mat{W} \Vec{e}(\bar{\Vec{x}})
    }_{\Vec{b}}
\end{align}

Solve the normal equations $\Mat{H}\Delta\Vec{x} = \Vec{b}$ for
$\Delta\Vec{x}$ using the Cholesky or QR-decompositon. Once
$\Delta\Vec{x}$ is found the best estimate $\bar{\Vec{x}}$ can be
updated via,

\begin{align}
  \bar{\Vec{x}}_{k + 1} = \bar{\Vec{x}}_{k} + \Delta\Vec{x}.
\end{align}



\section{Schur Complement}

Let $\Mat{M}$ be a matrix that consists of block matrices
$\Mat{A}$, $\Mat{B}$, $\Mat{C}$, $\Mat{D}$,

\begin{align}
  \Mat{M} =
  \begin{bmatrix}
    \Mat{A} & \Mat{B} \\
    \Mat{C} & \Mat{D}
  \end{bmatrix}
\end{align}

if $\Mat{A}$ is invertible, the Schur's complement of the block
$\Mat{A}$ of the matrix $\Mat{B}$ is the defined by

\begin{aligned}
  \Mat{M}/\Mat{A} &= \Mat{D} - \Mat{C} \Mat{A}^{-1} \Mat{B} \\
  \Mat{M}/\Mat{D} &= \Mat{A} - \Mat{B} \Mat{D}^{-1} \Mat{C}
\end{aligned}



\end{document}
