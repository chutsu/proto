\chapter{Rotations}

\section{Euler Angles}
Z-Y-X rotation sequence:

\begin{equation}
  \rot_{zyx} =
  \begin{bmatrix}
    c(\psi) c(\theta)
    & c(\psi) s(\theta) s(\phi) - s(\psi) c(\phi)
    & c(\psi) s(\theta) c(\phi) + s(\psi) s(\phi) \\
    s(\psi) c(\theta)
    & s(\psi) s(\theta) s(\phi) + c(\psi) c(\phi)
    & s(\psi) s(\theta) c(\phi) - c(\psi) s(\phi) \\
    -s(\theta) & c(\theta) s(\phi) & c(\theta) c(\phi)
  \end{bmatrix}
\end{equation}



% QUATERNIONS
\section{Quaternions}

A quaternion, $\Vec{q} \in \real^{4}$, generally has the following form
%
\begin{equation}
  \quat = q_{w} + q_{x} \mathbf{i} + q_{y} \mathbf{j} + q_{z} \mathbf{k},
\end{equation}
%
where $\{ q_{w}, q_{x}, q_{y}, q_{z} \} \in \real$ and $\{ \mathbf{i}, \mathbf{j},
\mathbf{k} \}$ are the imaginary numbers satisfying
%
\begin{equation}
\begin{split}
  &\mathbf{i}^{2}
  = \mathbf{j}^{2}
  = \mathbf{k}^{2}
  = \mathbf{ijk}
  = -1 \\
  \mathbf{ij} = -\mathbf{ji} &= \mathbf{k}, \enspace
  \mathbf{jk} = -\mathbf{kj} = \mathbf{i}, \enspace
  \mathbf{ki} = -\mathbf{ik} = \mathbf{j}
\end{split}
\end{equation}
%
corresponding to the Hamiltonian convention. The quaternion can be written as a
4 element vector consisting of a \textit{real} (\textit{scalar}) part, $q_{w}$,
and \textit{imaginary} (\textit{vector}) part $\quat_{v}$ as,
%
\begin{equation}
  \quat =
  \begin{bmatrix} q_{w} \\ \quat_{v} \end{bmatrix} =
  \begin{bmatrix} q_{w} \\ q_{x} \\ q_{y} \\ q_{z} \end{bmatrix}
\end{equation}
%
There are other quaternion conventions, for example, the JPL convention. A more
detailed discussion between Hamiltonian and JPL quaternion convention is
discussed in \cite{Sola2017}.


\subsection{Main Quaternion Properties}


\subsubsection{Sum}

Let $\Vec{p}$ and $\Vec{q}$ be two quaternions, the sum of both quaternions is,
%
\begin{equation}
  \Vec{p} \pm \Vec{q} =
  \begin{bmatrix} p_w \\ \Vec{p}_{v} \end{bmatrix}
  \pm
  \begin{bmatrix} q_w \\ \Vec{q}_{v} \end{bmatrix} =
  \begin{bmatrix} p_w \pm q_w \\ \Vec{p}_{v} \pm \Vec{q}_{v} \end{bmatrix}.
\end{equation}
%
The sum between two quaternions $\Vec{p}$ and $\Vec{q}$ is \textbf{commutative}
and \textbf{associative}.
%
\begin{equation}
  \Vec{p} + \Vec{q} = \Vec{q} + \Vec{p}
\end{equation}
%
\begin{equation}
  \Vec{p} + (\Vec{q} + \Vec{r}) = (\Vec{p} + \Vec{q}) + \Vec{r}
\end{equation}


\subsubsection{Product}

The quaternion multiplication (or product) of two quaternions $\Vec{p}$ and
$\Vec{q}$, denoted by $\otimes$ is defined as
%
\begin{align}
  \Vec{p} \otimes \Vec{q}
    &=
    (p_w + p_x \mathbf{i} + p_y \mathbf{j} + p_z \mathbf{k})
    (q_w + q_x \mathbf{i} + q_y \mathbf{j} + q_z \mathbf{k}) \\
    &=
    \begin{matrix}
      &(p_w q_w - p_x q_x - p_y q_y - p_z q_z)& \\
      &(p_w q_x + p_x q_w + p_y q_z - p_z q_y)& \mathbf{i}\\
      &(p_w q_y - p_y q_w + p_z q_x + p_x q_z)& \mathbf{j}\\
      &(p_w q_z + p_z q_w - p_x q_y + p_y q_x)& \mathbf{k}\\
    \end{matrix} \\
    &=
    \begin{bmatrix}
      \label{eq:quaternion_product}
      p_w q_w - p_x q_x - p_y q_y - p_z q_z \\
      p_w q_x + q_x p_w + p_y q_z - p_z q_y \\
      p_w q_y - p_y q_w + p_z q_x + p_x q_z \\
      p_w q_z + p_z q_w - p_x q_y + p_y q_x \\
    \end{bmatrix} \\
    &=
    \begin{bmatrix}
      \label{eq:quaternion_product_2}
      p_w q_w - \Transpose{\Vec{p}_{v}} \Vec{q}_{v} \\
      p_w \Vec{q}_{v} + q_w \Vec{p}_{v} + \Vec{p}_{v} \times \Vec{q}_{v}
    \end{bmatrix}.
\end{align}
%
The quaternion product is \textbf{not commutative} in the general
case\footnote{There are exceptions to the general non-commutative rule, where
either $\Vec{p}$ or $\Vec{q}$ is real such that $\Vec{p}_{v} \times \Vec{q}_{v}
= 0$, or when both $\Vec{p}_v$ and $\Vec{q}_v$ are parallel, $\Vec{p}_v ||
\Vec{q}_v$. Only in these cirmcumstances is the quaternion product
commutative.},
%
\begin{equation}
  {\Vec{p} \otimes \Vec{q} \neq \Vec{q} \otimes \Vec{p}} \enspace .
\end{equation}
%
The quaternion product is however \textbf{associative},
%
\begin{equation}
  \Vec{p} \otimes (\Vec{q} \otimes \Vec{r})
  = (\Vec{p} \otimes \Vec{q}) \otimes \Vec{r}
\end{equation}
%
and \textbf{distributive over the sum}
%
\begin{equation}
  \Vec{p} \otimes (\Vec{q} + \Vec{r}) =
  \Vec{p} \otimes \Vec{q} + \Vec{p} \otimes \Vec{r}
  \quad \text{and} \quad
  (\Vec{p} \otimes \Vec{q}) + \Vec{r} =
  \Vec{p} \otimes \Vec{r} + \Vec{q} \otimes \Vec{r}
\end{equation}

The quaternion product can alternatively be expressed in matrix form as
%
\begin{equation}
  \Vec{p} \otimes \Vec{q} = [\Vec{p}]_{L} \Vec{q}
  \quad \text{and} \quad
  \Vec{p} \otimes \Vec{q} = [\Vec{q}]_{R} \Vec{p} \enspace ,
\end{equation}
%
where $[\Vec{p}]_{L}$ and $[\Vec{q}]_{R}$ are the left and right
quaternion-product matrices which are derived from
\eqref{eq:quaternion_product},
%
\begin{equation}
  [\Vec{p}]_{L} =
  \begin{bmatrix}
    p_w & -p_x & -p_y & -p_z \\
    p_x & p_w & -p_z & p_y \\
    p_y & p_z & p_w & -p_x \\
    p_z & -p_y & p_x & p_w
  \end{bmatrix},
  \quad \text{and} \quad
  [\Vec{q}]_{R} =
  \begin{bmatrix}
    q_w & -q_x & -q_y & -q_z \\
    q_x & q_w & q_z & -q_y \\
    q_y & -q_z & q_w & q_x \\
    q_z & q_y & -q_x & q_w
  \end{bmatrix},
\end{equation}
%
or inspecting \eqref{eq:quaternion_product_2} a compact form can be derived as,
%
\begin{equation}
  [\Vec{p}]_{L} =
  \begin{bmatrix}
    0 & -\Transpose{\Vec{p}_{v}} \\
    \Vec{p}_w \I_{3 \times 3} + \Vec{p}_{v} &
    \Vec{p}_w \I_{3 \times 3} -\Skew{\Vec{p}_{v}}
  \end{bmatrix}
\end{equation}
%
and
%
\begin{equation}
  [\Vec{q}]_{R} =
  \begin{bmatrix}
    0 & -\Transpose{\Vec{q}_{v}} \\
    \Vec{q}_w \I_{3 \times 3} + \Vec{q}_{v} &
    \Vec{q}_w \I_{3 \times 3} -\Skew{\Vec{q}_{v}}
  \end{bmatrix},
\end{equation}
%
where $\Skew{\bullet}$ is the skew operator that produces a matrix cross
product matrix, and is defined as
%
\begin{equation}
  \Skew{\Vec{v}} =
  \begin{bmatrix}
    0 & -v_{3} & v_{2} \\
    v_{3} & 0 & -v_{1} \\
    -v_{2} & v_{1} & 0
  \end{bmatrix},
  \quad
  \Vec{v} \in \real^{3}
\end{equation}
%

\subsubsection{Conjugate}

The conjugate operator for quaternion, ${(\bullet)}^{\ast}$, is
defined as
%
\begin{equation}
  \quat^{\ast}
  =
  \begin{bmatrix}
    q_w \\
    - \Vec{q}_v
  \end{bmatrix}
  =
  \begin{bmatrix}
    q_w \\
    - q_x \\
    - q_y \\
    - q_z
  \end{bmatrix}.
\end{equation}
%
This has the properties
%
\begin{equation}
  \quat \otimes \quat^{-1}
  = \quat^{-1} \otimes \quat
  = q_{w}^{2} + q_{x}^{2} + q_{y}^{2} + q_{z}^{2}
  =
  \begin{bmatrix}
    q_{w}^{2} + q_{x}^{2} + q_{y}^{2} + q_{z}^{2} \\
    \Vec{0}
  \end{bmatrix},
\end{equation}
%
and
%
\begin{equation}
  (\Vec{p} \otimes \Vec{q})^{\ast}
  = \Vec{q}^{\ast} \otimes \Vec{p}^{\ast}.
\end{equation}


\subsubsection{Norm}

The norm of a quaternion is defined by
%
\begin{align}
  \Norm{\quat} &= \sqrt{\quat \otimes \quat^{\ast}} \\
    &= \sqrt{\quat^{\ast} \otimes \quat} \\
    &= \sqrt{q_{w}^{2} + q_{x}^{2} + q_{y}^{2} + q_{z}^{2}}
    \enspace \in \real,
\end{align}
%
and has the property
%
\begin{align}
  \Norm{\Vec{p} \otimes \Vec{q}} =
  \Norm{\Vec{q} \otimes \Vec{p}} =
  \Norm{\Vec{p}} \Norm{\Vec{q}}
\end{align}
